\documentclass[preprint,12pt]{elsarticle}

\usepackage{amssymb}

\usepackage{lineno}

\journal{Advanced Engineering Informatics}

\begin{document}

\begin{frontmatter}

%% Title, authors and addresses

\title{A framework for Process Optimization in Engineering Design Based on Hybrid Knowledge Structure Generation and Binary Matrix Factorization}

\author[ad1]{Shubin Zhao\fnref{label1}}
\ead{nickyzhao@outlook.com}

\author[ad2]{Gang Li\corref{cor2}\fnref{label2}}
\ead{gang.li@deakin.edu.au}

\author[ad1]{Cheng Xu\corref{cor3}\fnref{label3}}
\ead{xucheng62@njust.edu.cn}

\cortext[cor2]{Principal corresponding author}
\cortext[cor3]{Corresponding author}
\fntext[label1]{This is a specimen author footnote}
\fntext[label2]{Another author footnote, but a little more longer.}
\fntext[label3]{Yet another author footnote. Indeed, you can have any number of author footnotes.}

\address[ad1]{Nanjing University of Science and Technology, China}
\address[ad2]{Deakin University, Australia}

%% Text of abstract & keywords

\begin{abstract}
Issue-answer decomposing is not a novel method in engineering design area. 
Decades of study have shown that it is significant for the reuse of implicit knowledge structure in the design process.
However, the establishment of issue-based knowledge structure remains highly human-involved,
and limited efforts have been made on the further processing of decomposition results.
Some have paid attention to the utilization of Design Matrix (DM), which is derived from the structure, 
in order to solve decoupling problems, or put forward some modular design methodolgy.
However, the current research and application of DM is mostly limited to the conversion of it 
to another matrix form, Design Structure Matrix (DSM), which reflects the interaction relationship of system components
but is insufficient to  handle the situation when internal construction of the system is not clear.
This paper attemps to propose a framework of process optimization in engineering design field, 
based on the semi-automatic generation of issue-based Hybrid Knowledge Structure (HKS)
and a binary matrix factorization method applied directly on DM.
A case study of electro-hydraulic dram brake design is used 
to exemplify how process optimization can be accomplished by matrix factorization.
We also illustrate how HKS works in the process.
Lastly, in order to highlight the value of our method,
we briefly discuss some potential research interests and further applications.
\end{abstract}

\begin{keyword}
engineering design \sep feature extraction \sep design informatics \sep knowledge structure \sep modularization \sep binary matrix factorization
\end{keyword}

\end{frontmatter}

%% Start line numbering here if you want

\linenumbers

%% main text

\section{Introduction}\label{sec-intro}

%背景介绍及现状分析
The extraction and representation of design knowledge in engineering design field 
remains a time-consuming and highly human-involved task. 
One of the challenges of most knowledge systems 
is to offer a comprehensive representation scheme
to deal with diversities of knowledge types. 
In engineering design area,
there might exist texts, data sheets, graphs, engineering drawings, theoretical formulas and so on, 
some of which are organized in a logically structured way, but some are not.

Regli~{et~al.}~\cite{Regli2000A} decribes the representation of structured knowledge in design documents
as two types: Process-oriented and Feature-oriented.
The former originates from Issue-Based Information System (IBIS) framework
for ongoing design process recording.
This method, and a few other IBIS-based extending frameworks,
generally take in the form of a graph, using nodes and links
to represent knowledge units (issues, answers, arguments, pros and cons, etc.) and their relations.
These approaches usually reflect the decision-makings, including their reasons, 
throughout the design process.
While the latter seeks a more formal framework 
to model the rationales that exist in design reports, patent documents, manuals, etc.
In engineering design area, rationales can be considered as objective causalities 
among functions, design parameters, theoretical derivations, artifacts and manufacturing information.
A well-organized structuring of these knowledge units 
can function as an experienced domain expert 
that supports the decision-makings in a design or redesign process.
These approaches in essence have an explicit, logical structure 
and can easily be understood and processed by computers, 
making it possible that design rationales provide timely, point-to-point support in a design process.
However, the ease of reuse has also brought with it some disadvantages.
Establishment of such knowledge systems is inevitably time-consuming,
and relies heavily on human efforts and expertise.

As for unstructured design knowledge, 
the extracting appoach usually develops 
in sync with advancing Text Mining techniques.
Design documents are first converted into plain texts,
then tokenization and POS tagging can be done on them.
Ontology-based entity and concept recognition
has been a dominant approach for text feature extraction,
both in industry and academia.
Lim~{et~al.}~\cite{Lim2010Multi} proposes an ontology-based 
entity-concept-facet framework for product information extraction,
based on which possible relations are discovered by semantic tags.
The dicovery of relations between entities and concepts
are usually based on an automatic procedure,
thus having limited precision compared to those manually intensive,
depending on the performance of adopted algorithm.
Furtherly, knowledge retrieval based on these methods 
usually provide the designer with a ranked list of knowledge units,
based on the matching degree between them and the query.
In other words, the retrieving process is keyword-based, but not rule-based,
thus having no point-to-point logic, 
and basically relying on the designer's mastery of the design issue
to provide keyword-based queries.

Combinations of above-mentioned approaches have been put forward 
to make use of their individual advantages. 
A typical example is Issue, Solution and Artifact Layer (ISAL) model
for design rationale representation~\cite{Liang2012Learning}.
The model uses a issue-based triple-layer structure 
to extract text features as issue-solution pairs and artifact information.
The causal relations between the pairs and artifact information
are derived automatically from discrete knowledge units, using text mining techniques, 
which forms a issue-based, feature-oriented procedure of design rationale extraction.
The method offers a meaningful hybrid prototype for an integration
of different knowledge representation schemes.

Above-mentioned researches have provided instructive insight into 
the extraction of design knowledge units,
and the establishment of corresponding representation models.
After that, knowledge retrieval can be conducted 
with higher precision, multiple options, 
or even unintentional, but highly related answers.
However, another critical issue we have to consider in engineering design process 
is that, solutions of different issues (or their sub-issues) 
usually proceed in a sequential way, but not a parallel way,
due to the coupling problem that may occur
when we attempt to solve several problems at the same time.
In other words, design process needs to be optimized
once we have obtained the issue-based, decomposed knowledge struture.

Axiomatic Design (AD)~\cite{Suh2001Axiomatic} provides a theoretical framework
for the generation of issue-based, tree-like decomposition structure, 
based on the zigzag mapping between issue domain and answer domain.
Then DM, a matrix with issues as its rows and answers as its columns, 
can be naturally generated according to the structure,
which reflects the correspondence between issues and answers.
AD views a design process as a favorable one, 
only if DM is diagonal, or triangular.
In this sense, DM is of great significance to design process optimization.
However, how to deal with DM was not discussed in AD.
Studies addressing this issue 
have naturally lead to another research area: Modular Design.
Researchers have been sought to gather those closely-related answers
into clusters, in order to minimize the correlation between clusters, i.e., modules.
In this way, the problem has changed 
from finding a sequence of individual answers to designing a set of modules,
which can be recognized as common modules, and archived for future reuse. 
Therefore, the task is to find valid clustering algorithm
to guarantee the effectiveness of module partitioning.
Recent advances such as~\cite{Xiao2012New}~and~\cite{Beek2010Modular}
have introduced DSM to assist in the analysis of 
interaction relationship between answers to the issues. 
Distinguished from DM, 
DSM takes answers as both its rows and columns.
How they are related to each other and how much they are related
are considered in the matrix, 
then clustering algorithm, e.g., k-means method, 
is utilized to generate modules of the answers.
Their approaches have offered us some insight
into the establishment of common module base, 
especially in engineering design field,
where reuse of product module is quite frequent.
However, the conversion of DM into DSM itself does not generate modules, 
which means specific clustering algorithms have to be adopted for modularization.
Elements in DSM have to be decided by domain experts,
in that they explain the specific interaction relationship between system components,
regularly in the form of a value in the range [0,1],
representing the intensity of relations.

In this paper, we propose a methodology for building an issue-based hybrid HKS,
in an attempt to combine ontology-based Named Entity Recognition (NER) 
with issue-based knowledge representation framework.
Based on the structure, matrix factorization is conducted directly 
on DM, in order to extract eigenvectors as fundamental functional modules.
Particularly, we introduce Binary Matrix Factorization (BMF)~\cite{Zhang2007Binary}
to deal with binary DM, considering there is usually a relation (1) or no relation (0) 
from one component to another in concept design stage.
The rest of the paper is discussed as follows:...

\section{Related work}\label{sec-works}

\subsection{Representation Schemes for Design Knowledge}\label{subsec-representation}
%具体优缺点展开
As mentioned in Section \ref{sec-intro}, we have

\subsection{Design Matrix Processing and Modular Design}\label{subsec-modular}
%由引入

\subsection{Binary Matrix Factorization}\label{subsec-bmf}

\section{Methodological approach}\label{sec-method}

\subsection{Framework}\label{subsec-framework}

\subsection{Knowledge Structure Generation}\label{subsec-structure}

\subsection{Feature Extraction as Modules}\label{subsec-feature}

\section{Case Study and Discussion}\label{sec-case}

\section{Conclusion and Future Work}\label{sec-conclusion}


%% References with bibTeX database:

\bibliographystyle{unsrt}
\bibliography{sample}

\end{document}